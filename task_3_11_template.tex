\documentclass[12pt,a4paper]{article}
\usepackage[utf8]{inputenc}
\usepackage[russian]{babel}
\usepackage{amsmath}
\usepackage{amsfonts}
\usepackage{amssymb}
\usepackage{graphicx}
\usepackage{booktabs}
\usepackage{diagbox}
\usepackage{float}
\usepackage[left=2cm,right=2cm,top=2cm,bottom=2cm]{geometry}

\title{Численное решение задачи 3.11}
\author{}
\date{}

\begin{document}

\maketitle

\section{Постановка задачи}

Задача 3.11 представляет собой численное решение системы уравнений газовой динамики:

\begin{aligned}
& G_t + 0.5(V \hat{G}_{\mathring{x}} + (V \hat{G})_{\mathring{x}} + (2 - G)V_{\mathring{x}}) = 0, \quad x \in \omega_h, \\
& G_{t,0} + 0.5\left((V \hat{G})_{x,0} + (2 - G_0)V_{x,0}\right) \\
& \quad - 0.5h\left((GV)_{x\bar{x},1} - 0.5(GV)_{x\bar{x},2} + (2 - G_0)\left(V_{x\bar{x},1} - 0.5V_{x\bar{x},2}\right)\right) = 0, \\
& G_{t,M} + 0.5\left((V \hat{G})_{\bar{x},M} + (2 - G_M)V_{\bar{x},M}\right) \\
& \quad + 0.5h\left((GV)_{x\bar{x},M-1} - 0.5(GV)_{x\bar{x},M-2} + (2 - G_M)\left(V_{x\bar{x},M-1} - 0.5V_{x\bar{x},M-2}\right)\right) = 0, \\
& V_t + \frac{1}{3}(V \hat{V}_{\mathring{x}} + (V \hat{V})_{\mathring{x}}) + \tilde{p}'(\hat{G})\hat{G}_{\mathring{x}} = \tilde{\mu} \hat{V}_{x\bar{x}} - (\tilde{\mu} - \mu e^{-\hat{G}})V_{x\bar{x}} + f, \quad x \in \omega_h,
\end{aligned}

где:
\begin{itemize}
    \item $G = \ln(\rho)$ -- логарифм плотности
    \item $V$ -- скорость
    \item $p(\rho)$ -- давление (два варианта: $p(\rho) = C_\rho \rho$ или $p(\rho) = C_\rho \rho^{1.4}$)
    \item $\mu$ -- коэффициент вязкости
\end{itemize}

\subsection{Тестовое решение}

Для проверки точности используется аналитическое решение:
\begin{align*}
\rho(x,t) &= e^t \cdot (\cos(3\pi x) + 1.5) \\
u(x,t) &= \cos(2\pi t) \cdot \sin(4\pi x)
\end{align*}

\subsection{Вычисляемые нормы невязок}

Для оценки точности вычисляются три нормы невязок:

\begin{enumerate}
    \item \textbf{C-норма} (равномерная): $||e||_C = \max_{i} |u_i^{num} - u_i^{exact}|$
    \item \textbf{L2-норма} (интегральная): $||e||_{L_2} = \sqrt{h \sum_{i} (u_i^{num} - u_i^{exact})^2}$
    \item \textbf{W2-норма} (Соболева): $||e||_{W_2} = \sqrt{||e||_{L_2}^2 + \frac{1}{h}\sum_{i} (e_i - e_{i-1})^2}$
\end{enumerate}

\section{Результаты для G (логарифм плотности)}

В таблицах ниже представлены результаты вычислений для различных параметров. Каждая ячейка содержит 4 строки:
\begin{enumerate}
    \item Невязка в C-норме
    \item Невязка в L2-норме
    \item Невязка в W2-норме
    \item Произведение $\tau \times h$
\end{enumerate}

% Вставка таблиц для G
% Раскомментируйте следующую строку после генерации таблиц:
% \begin{tabular}{ |l|l|l|l|l| }
\hline
\multicolumn{5}{|c|}{$\mu = 0.1, p(\rho) = 1\rho^{1.4}$} \\
\hline
$\tau\setminus h$ & $0.1$ & $0.01$ & $0.001$ & $0.0001$\\
\hline
$0.1$ & $8.930616e+00$ & $6.141342e+02$ & $2.318400e+03$ & $5.336656e+04$ \\
& $5.770499e+00$ & $6.845855e+01$ & $1.513960e+02$ & $1.447461e+03$ \\
& $7.067389e+00$ & $8.384426e+01$ & $1.854215e+02$ & $1.772770e+03$ \\
& $9.102342e+00$ & $8.913374e+01$ & $2.380101e+02$ & $2.288630e+03$ \\
\hline
$1.000000e-02$ & $1.417561e+02$ & $3.372217e-02$ & $2.809707e-02$ & $2.807240e-02$ \\
& $4.705630e+01$ & $1.336401e-02$ & $1.184328e-02$ & $1.181994e-02$ \\
& $5.763196e+01$ & $1.636751e-02$ & $1.450500e-02$ & $1.447641e-02$ \\
& $7.398328e+01$ & $2.085954e-02$ & $1.871042e-02$ & $1.868740e-02$ \\
\hline
$1.000000e-03$ & $4.255965e+00$ & $1.058766e-02$ & $2.555489e-03$ & $2.510216e-03$ \\
& $1.526775e+00$ & $3.112507e-03$ & $1.319475e-03$ & $1.314785e-03$ \\
& $1.869910e+00$ & $3.812027e-03$ & $1.616020e-03$ & $1.610277e-03$ \\
& $2.004067e+00$ & $4.806066e-03$ & $2.084724e-03$ & $2.078712e-03$ \\
\hline
$1.000000e-04$ & $4.656531e+00$ & $8.546182e-03$ & $3.274727e-04$ & $2.480697e-04$ \\
& $1.637069e+00$ & $2.652310e-03$ & $1.383470e-04$ & $1.331346e-04$ \\
& $2.004992e+00$ & $3.248404e-03$ & $1.694398e-04$ & $1.630559e-04$ \\
& $2.128769e+00$ & $4.105667e-03$ & $2.185014e-04$ & $2.104900e-04$ \\
\hline
\end{tabular}


\begin{tabular}{ |l|l|l|l|l| }
\hline
\multicolumn{5}{|c|}{$\mu = 1.000000e-01, p(\rho) = 1.000000e+00\rho$} \\
\hline
$\tau\setminus h$ & $0.1$ & $0.01$ & $0.001$ & $0.0001$\\
\hline
$1.000000e-01$ & $9.417089e+00$ & $7.134958e+00$ & $1.392707e+01$ & $2.709828e+01$ \\
& $5.545571e+00$ & $2.299721e+00$ & $2.195744e+00$ & $2.176660e+00$ \\
& $6.791910e+00$ & $2.816572e+00$ & $2.689226e+00$ & $2.665853e+00$ \\
& $8.634553e+00$ & $3.603591e+00$ & $3.467529e+00$ & $3.441131e+00$ \\
\hline
$1.000000e-02$ & $2.922101e+02$ & $1.954389e-01$ & $1.815556e-01$ & $1.814323e-01$ \\
& $1.094415e+02$ & $9.356493e-02$ & $8.908686e-02$ & $8.902028e-02$ \\
& $1.340380e+02$ & $1.145932e-01$ & $1.091087e-01$ & $1.090271e-01$ \\
& $1.696682e+02$ & $1.478190e-01$ & $1.408415e-01$ & $1.407517e-01$ \\
\hline
$1.000000e-03$ & $7.989753e+00$ & $1.854913e-02$ & $1.390458e-02$ & $1.385912e-02$ \\
& $2.823273e+00$ & $1.107670e-02$ & $7.464454e-03$ & $7.429236e-03$ \\
& $3.457790e+00$ & $1.356614e-02$ & $9.142052e-03$ & $9.098919e-03$ \\
& $3.680159e+00$ & $1.751357e-02$ & $1.180158e-02$ & $1.174658e-02$ \\
\hline
$1.000000e-04$ & $2.171788e+00$ & $6.108071e-03$ & $1.385983e-03$ & $1.343475e-03$ \\
& $1.326248e+00$ & $4.616529e-03$ & $7.622230e-04$ & $7.289514e-04$ \\
& $1.624315e+00$ & $5.654071e-03$ & $9.335287e-04$ & $8.927795e-04$ \\
& $2.071433e+00$ & $7.279154e-03$ & $1.205131e-03$ & $1.152566e-03$ \\
\hline
\end{tabular}


\begin{tabular}{ |l|l|l|l|l| }
\hline
\multicolumn{5}{|c|}{$\mu = 1.000000e-01, p(\rho) = 1.000000e+01\rho^{1.4}$} \\
\hline
$\tau\setminus h$ & $0.1$ & $0.01$ & $0.001$ & $0.0001$\\
\hline
$1.000000e-01$ & $8.930616e+00$ & $6.141342e+02$ & $2.318400e+03$ & $5.336656e+04$ \\
& $5.770499e+00$ & $6.845855e+01$ & $1.513960e+02$ & $1.447461e+03$ \\
& $7.067389e+00$ & $8.384426e+01$ & $1.854215e+02$ & $1.772770e+03$ \\
& $9.102342e+00$ & $8.913374e+01$ & $2.380101e+02$ & $2.288630e+03$ \\
\hline
$1.000000e-02$ & $1.417561e+02$ & $3.372217e-02$ & $2.809707e-02$ & $2.807240e-02$ \\
& $4.705630e+01$ & $1.336401e-02$ & $1.184328e-02$ & $1.181994e-02$ \\
& $5.763196e+01$ & $1.636751e-02$ & $1.450500e-02$ & $1.447641e-02$ \\
& $7.398328e+01$ & $2.085954e-02$ & $1.871042e-02$ & $1.868740e-02$ \\
\hline
$1.000000e-03$ & $4.255965e+00$ & $1.058766e-02$ & $2.555489e-03$ & $2.510216e-03$ \\
& $1.526775e+00$ & $3.112507e-03$ & $1.319475e-03$ & $1.314785e-03$ \\
& $1.869910e+00$ & $3.812027e-03$ & $1.616020e-03$ & $1.610277e-03$ \\
& $2.004067e+00$ & $4.806066e-03$ & $2.084724e-03$ & $2.078712e-03$ \\
\hline
$1.000000e-04$ & $4.656531e+00$ & $8.546182e-03$ & $3.274727e-04$ & $2.480697e-04$ \\
& $1.637069e+00$ & $2.652310e-03$ & $1.383470e-04$ & $1.331346e-04$ \\
& $2.004992e+00$ & $3.248404e-03$ & $1.694398e-04$ & $1.630559e-04$ \\
& $2.128769e+00$ & $4.105667e-03$ & $2.185014e-04$ & $2.104900e-04$ \\
\hline
\end{tabular}


\begin{tabular}{ |l|l|l|l|l| }
\hline
\multicolumn{5}{|c|}{$\mu = 1.000000e-01, p(\rho) = 1.000000e+01\rho$} \\
\hline
$\tau\setminus h$ & $0.1$ & $0.01$ & $0.001$ & $0.0001$\\
\hline
$1.000000e-01$ & $3.331907e+03$ & $2.238448e+01$ & $6.546037e+03$ & $6.796516e+02$ \\
& $1.627620e+03$ & $5.610216e+00$ & $2.694481e+02$ & $1.118948e+01$ \\
& $1.993419e+03$ & $6.871083e+00$ & $3.300052e+02$ & $1.370426e+01$ \\
& $2.347915e+03$ & $8.583452e+00$ & $4.235235e+02$ & $1.769073e+01$ \\
\hline
$1.000000e-02$ & $6.143129e+01$ & $1.009663e-02$ & $1.069107e-02$ & $1.069705e-02$ \\
& $2.752249e+01$ & $7.239306e-03$ & $7.295930e-03$ & $7.292684e-03$ \\
& $3.370803e+01$ & $8.866303e-03$ & $8.935653e-03$ & $8.931677e-03$ \\
& $4.263650e+01$ & $1.140173e-02$ & $1.153387e-02$ & $1.153055e-02$ \\
\hline
$1.000000e-03$ & $1.574454e+03$ & $3.817860e-03$ & $1.031406e-03$ & $1.037394e-03$ \\
& $7.464840e+02$ & $1.109296e-03$ & $7.246988e-04$ & $7.255395e-04$ \\
& $9.142525e+02$ & $1.358604e-03$ & $8.875712e-04$ & $8.886008e-04$ \\
& $1.168332e+03$ & $1.711894e-03$ & $1.145633e-03$ & $1.147159e-03$ \\
\hline
$1.000000e-04$ & $0.000000e+00$ & $3.218861e-03$ & $9.922216e-05$ & $1.034054e-04$ \\
& $-nan$ & $9.417588e-04$ & $7.187321e-05$ & $7.253863e-05$ \\
& $-nan$ & $1.153414e-03$ & $8.802635e-05$ & $8.884131e-05$ \\
& $-nan$ & $1.453844e-03$ & $1.135982e-04$ & $1.146916e-04$ \\
\hline
\end{tabular}


\begin{tabular}{ |l|l|l|l|l| }
\hline
\multicolumn{5}{|c|}{$\mu = 1.000000e-01, p(\rho) = 1.000000e+02\rho^{1.4}$} \\
\hline
$\tau\setminus h$ & $0.1$ & $0.01$ & $0.001$ & $0.0001$\\
\hline
$1.000000e-01$ & $8.930616e+00$ & $6.141342e+02$ & $2.318400e+03$ & $5.336656e+04$ \\
& $5.770499e+00$ & $6.845855e+01$ & $1.513960e+02$ & $1.447461e+03$ \\
& $7.067389e+00$ & $8.384426e+01$ & $1.854215e+02$ & $1.772770e+03$ \\
& $9.102342e+00$ & $8.913374e+01$ & $2.380101e+02$ & $2.288630e+03$ \\
\hline
$1.000000e-02$ & $1.417561e+02$ & $3.372217e-02$ & $2.809707e-02$ & $2.807240e-02$ \\
& $4.705630e+01$ & $1.336401e-02$ & $1.184328e-02$ & $1.181994e-02$ \\
& $5.763196e+01$ & $1.636751e-02$ & $1.450500e-02$ & $1.447641e-02$ \\
& $7.398328e+01$ & $2.085954e-02$ & $1.871042e-02$ & $1.868740e-02$ \\
\hline
$1.000000e-03$ & $4.255965e+00$ & $1.058766e-02$ & $2.555489e-03$ & $2.510216e-03$ \\
& $1.526775e+00$ & $3.112507e-03$ & $1.319475e-03$ & $1.314785e-03$ \\
& $1.869910e+00$ & $3.812027e-03$ & $1.616020e-03$ & $1.610277e-03$ \\
& $2.004067e+00$ & $4.806066e-03$ & $2.084724e-03$ & $2.078712e-03$ \\
\hline
$1.000000e-04$ & $4.656531e+00$ & $8.546182e-03$ & $3.274727e-04$ & $2.480697e-04$ \\
& $1.637069e+00$ & $2.652310e-03$ & $1.383470e-04$ & $1.331346e-04$ \\
& $2.004992e+00$ & $3.248404e-03$ & $1.694398e-04$ & $1.630559e-04$ \\
& $2.128769e+00$ & $4.105667e-03$ & $2.185014e-04$ & $2.104900e-04$ \\
\hline
\end{tabular}


\begin{tabular}{ |l|l|l|l|l| }
\hline
\multicolumn{5}{|c|}{$\mu = 1.000000e-01, p(\rho) = 1.000000e+02\rho$} \\
\hline
$\tau\setminus h$ & $0.1$ & $0.01$ & $0.001$ & $0.0001$\\
\hline
$1.000000e-01$ & $0.000000e+00$ & $0.000000e+00$ & $0.000000e+00$ & $0.000000e+00$ \\
& $-nan$ & $-nan$ & $-nan$ & $-nan$ \\
& $-nan$ & $-nan$ & $-nan$ & $-nan$ \\
& $-nan$ & $-nan$ & $-nan$ & $-nan$ \\
\hline
$1.000000e-02$ & $0.000000e+00$ & $0.000000e+00$ & $0.000000e+00$ & $0.000000e+00$ \\
& $-nan$ & $-nan$ & $-nan$ & $-nan$ \\
& $-nan$ & $-nan$ & $-nan$ & $-nan$ \\
& $-nan$ & $-nan$ & $-nan$ & $-nan$ \\
\hline
$1.000000e-03$ & $0.000000e+00$ & $3.541933e-03$ & $1.271272e-04$ & $1.050977e-04$ \\
& $-nan$ & $1.216048e-03$ & $8.655042e-05$ & $8.221857e-05$ \\
& $-nan$ & $1.489348e-03$ & $1.060022e-04$ & $1.006968e-04$ \\
& $-nan$ & $1.889835e-03$ & $1.367892e-04$ & $1.299956e-04$ \\
\hline
$1.000000e-04$ & $0.000000e+00$ & $3.426800e-03$ & $4.356928e-05$ & $1.040437e-05$ \\
& $-nan$ & $1.176136e-03$ & $1.601396e-05$ & $8.247628e-06$ \\
& $-nan$ & $1.440466e-03$ & $1.961301e-05$ & $1.010124e-05$ \\
& $-nan$ & $1.827788e-03$ & $2.528278e-05$ & $1.304029e-05$ \\
\hline
\end{tabular}


\begin{tabular}{ |l|l|l|l|l| }
\hline
\multicolumn{5}{|c|}{$\mu = 1.000000e-02, p(\rho) = 1.000000e+00\rho^{1.4}$} \\
\hline
$\tau\setminus h$ & $0.1$ & $0.01$ & $0.001$ & $0.0001$\\
\hline
$1.000000e-01$ & $8.956171e+01$ & $0.000000e+00$ & $0.000000e+00$ & $0.000000e+00$ \\
& $5.923464e+01$ & $-nan$ & $-nan$ & $-nan$ \\
& $7.254733e+01$ & $-nan$ & $-nan$ & $-nan$ \\
& $8.927333e+01$ & $-nan$ & $-nan$ & $-nan$ \\
\hline
$1.000000e-02$ & $0.000000e+00$ & $9.652502e-02$ & $9.221856e-02$ & $9.218052e-02$ \\
& $-nan$ & $4.807634e-02$ & $4.665577e-02$ & $4.661476e-02$ \\
& $-nan$ & $5.888125e-02$ & $5.714142e-02$ & $5.709119e-02$ \\
& $-nan$ & $7.540003e-02$ & $7.371161e-02$ & $7.369865e-02$ \\
\hline
$1.000000e-03$ & $0.000000e+00$ & $1.763202e-02$ & $9.065044e-03$ & $8.976927e-03$ \\
& $-nan$ & $5.977205e-03$ & $4.490544e-03$ & $4.477721e-03$ \\
& $-nan$ & $7.320552e-03$ & $5.499771e-03$ & $5.484066e-03$ \\
& $-nan$ & $9.284857e-03$ & $7.094384e-03$ & $7.079329e-03$ \\
\hline
$1.000000e-04$ & $0.000000e+00$ & $9.563756e-03$ & $9.828557e-04$ & $8.949533e-04$ \\
& $-nan$ & $2.841179e-03$ & $4.551424e-04$ & $4.458879e-04$ \\
& $-nan$ & $3.479720e-03$ & $5.574333e-04$ & $5.460989e-04$ \\
& $-nan$ & $4.389316e-03$ & $7.189718e-04$ & $7.049538e-04$ \\
\hline
\end{tabular}


\begin{tabular}{ |l|l|l|l|l| }
\hline
\multicolumn{5}{|c|}{$\mu = 1.000000e-02, p(\rho) = 1.000000e+00\rho$} \\
\hline
$\tau\setminus h$ & $0.1$ & $0.01$ & $0.001$ & $0.0001$\\
\hline
$1.000000e-01$ & $1.304231e+01$ & $4.443920e+03$ & $1.573513e+02$ & $1.333228e+03$ \\
& $5.896709e+00$ & $6.059468e+02$ & $1.113890e+01$ & $3.345728e+01$ \\
& $7.221964e+00$ & $7.421303e+02$ & $1.364231e+01$ & $4.097663e+01$ \\
& $9.109607e+00$ & $9.577239e+02$ & $1.757919e+01$ & $5.288217e+01$ \\
\hline
$1.000000e-02$ & $4.436713e+02$ & $1.443771e-01$ & $1.480135e-01$ & $1.480710e-01$ \\
& $2.517379e+02$ & $5.456098e-02$ & $5.683369e-02$ & $5.683881e-02$ \\
& $3.083147e+02$ & $6.682329e-02$ & $6.960677e-02$ & $6.961304e-02$ \\
& $3.724856e+02$ & $8.611867e-02$ & $8.984880e-02$ & $8.986874e-02$ \\
\hline
$1.000000e-03$ & $0.000000e+00$ & $1.408418e-02$ & $1.505924e-02$ & $1.507601e-02$ \\
& $-nan$ & $4.819858e-03$ & $6.095200e-03$ & $6.117189e-03$ \\
& $-nan$ & $5.903097e-03$ & $7.465065e-03$ & $7.491995e-03$ \\
& $-nan$ & $7.573882e-03$ & $9.635677e-03$ & $9.671959e-03$ \\
\hline
$1.000000e-04$ & $0.000000e+00$ & $5.522661e-03$ & $1.495841e-03$ & $1.507614e-03$ \\
& $-nan$ & $3.617128e-03$ & $5.924784e-04$ & $6.156868e-04$ \\
& $-nan$ & $4.430058e-03$ & $7.256348e-04$ & $7.540593e-04$ \\
& $-nan$ & $5.708973e-03$ & $9.365961e-04$ & $9.734694e-04$ \\
\hline
\end{tabular}


\begin{tabular}{ |l|l|l|l|l| }
\hline
\multicolumn{5}{|c|}{$\mu = 1.000000e-02, p(\rho) = 1.000000e+01\rho^{1.4}$} \\
\hline
$\tau\setminus h$ & $0.1$ & $0.01$ & $0.001$ & $0.0001$\\
\hline
$1.000000e-01$ & $8.956171e+01$ & $0.000000e+00$ & $0.000000e+00$ & $0.000000e+00$ \\
& $5.923464e+01$ & $-nan$ & $-nan$ & $-nan$ \\
& $7.254733e+01$ & $-nan$ & $-nan$ & $-nan$ \\
& $8.927333e+01$ & $-nan$ & $-nan$ & $-nan$ \\
\hline
$1.000000e-02$ & $0.000000e+00$ & $9.652502e-02$ & $9.221856e-02$ & $9.218052e-02$ \\
& $-nan$ & $4.807634e-02$ & $4.665577e-02$ & $4.661476e-02$ \\
& $-nan$ & $5.888125e-02$ & $5.714142e-02$ & $5.709119e-02$ \\
& $-nan$ & $7.540003e-02$ & $7.371161e-02$ & $7.369865e-02$ \\
\hline
$1.000000e-03$ & $0.000000e+00$ & $1.763202e-02$ & $9.065044e-03$ & $8.976927e-03$ \\
& $-nan$ & $5.977205e-03$ & $4.490544e-03$ & $4.477721e-03$ \\
& $-nan$ & $7.320552e-03$ & $5.499771e-03$ & $5.484066e-03$ \\
& $-nan$ & $9.284857e-03$ & $7.094384e-03$ & $7.079329e-03$ \\
\hline
$1.000000e-04$ & $0.000000e+00$ & $9.563756e-03$ & $9.828557e-04$ & $8.949533e-04$ \\
& $-nan$ & $2.841179e-03$ & $4.551424e-04$ & $4.458879e-04$ \\
& $-nan$ & $3.479720e-03$ & $5.574333e-04$ & $5.460989e-04$ \\
& $-nan$ & $4.389316e-03$ & $7.189718e-04$ & $7.049538e-04$ \\
\hline
\end{tabular}


\begin{tabular}{ |l|l|l|l|l| }
\hline
\multicolumn{5}{|c|}{$\mu = 1.000000e-02, p(\rho) = 1.000000e+01\rho$} \\
\hline
$\tau\setminus h$ & $0.1$ & $0.01$ & $0.001$ & $0.0001$\\
\hline
$1.000000e-01$ & $0.000000e+00$ & $0.000000e+00$ & $6.205870e+01$ & $1.374979e+02$ \\
& $-nan$ & $-nan$ & $1.712570e+01$ & $1.432379e+01$ \\
& $-nan$ & $-nan$ & $2.097461e+01$ & $1.754299e+01$ \\
& $-nan$ & $-nan$ & $2.705046e+01$ & $2.264757e+01$ \\
\hline
$1.000000e-02$ & $0.000000e+00$ & $0.000000e+00$ & $0.000000e+00$ & $0.000000e+00$ \\
& $-nan$ & $-nan$ & $-nan$ & $-nan$ \\
& $-nan$ & $-nan$ & $-nan$ & $-nan$ \\
& $-nan$ & $-nan$ & $-nan$ & $-nan$ \\
\hline
$1.000000e-03$ & $0.000000e+00$ & $2.631327e-03$ & $1.352473e-03$ & $1.361042e-03$ \\
& $-nan$ & $1.245135e-03$ & $7.764690e-04$ & $7.827724e-04$ \\
& $-nan$ & $1.524973e-03$ & $9.509764e-04$ & $9.586965e-04$ \\
& $-nan$ & $1.966843e-03$ & $1.227513e-03$ & $1.237652e-03$ \\
\hline
$1.000000e-04$ & $0.000000e+00$ & $2.863309e-03$ & $1.275354e-04$ & $1.359539e-04$ \\
& $-nan$ & $1.371176e-03$ & $7.230471e-05$ & $7.804374e-05$ \\
& $-nan$ & $1.679340e-03$ & $8.855483e-05$ & $9.558367e-05$ \\
& $-nan$ & $2.163841e-03$ & $1.143094e-04$ & $1.233961e-04$ \\
\hline
\end{tabular}


\begin{tabular}{ |l|l|l|l|l| }
\hline
\multicolumn{5}{|c|}{$\mu = 1.000000e-02, p(\rho) = 1.000000e+02\rho^{1.4}$} \\
\hline
$\tau\setminus h$ & $0.1$ & $0.01$ & $0.001$ & $0.0001$\\
\hline
$1.000000e-01$ & $8.956171e+01$ & $0.000000e+00$ & $0.000000e+00$ & $0.000000e+00$ \\
& $5.923464e+01$ & $-nan$ & $-nan$ & $-nan$ \\
& $7.254733e+01$ & $-nan$ & $-nan$ & $-nan$ \\
& $8.927333e+01$ & $-nan$ & $-nan$ & $-nan$ \\
\hline
$1.000000e-02$ & $0.000000e+00$ & $9.652502e-02$ & $9.221856e-02$ & $9.218052e-02$ \\
& $-nan$ & $4.807634e-02$ & $4.665577e-02$ & $4.661476e-02$ \\
& $-nan$ & $5.888125e-02$ & $5.714142e-02$ & $5.709119e-02$ \\
& $-nan$ & $7.540003e-02$ & $7.371161e-02$ & $7.369865e-02$ \\
\hline
$1.000000e-03$ & $0.000000e+00$ & $1.763202e-02$ & $9.065044e-03$ & $8.976927e-03$ \\
& $-nan$ & $5.977205e-03$ & $4.490544e-03$ & $4.477721e-03$ \\
& $-nan$ & $7.320552e-03$ & $5.499771e-03$ & $5.484066e-03$ \\
& $-nan$ & $9.284857e-03$ & $7.094384e-03$ & $7.079329e-03$ \\
\hline
$1.000000e-04$ & $0.000000e+00$ & $9.563756e-03$ & $9.828557e-04$ & $8.949533e-04$ \\
& $-nan$ & $2.841179e-03$ & $4.551424e-04$ & $4.458879e-04$ \\
& $-nan$ & $3.479720e-03$ & $5.574333e-04$ & $5.460989e-04$ \\
& $-nan$ & $4.389316e-03$ & $7.189718e-04$ & $7.049538e-04$ \\
\hline
\end{tabular}


\begin{tabular}{ |l|l|l|l|l| }
\hline
\multicolumn{5}{|c|}{$\mu = 1.000000e-02, p(\rho) = 1.000000e+02\rho$} \\
\hline
$\tau\setminus h$ & $0.1$ & $0.01$ & $0.001$ & $0.0001$\\
\hline
$1.000000e-01$ & $0.000000e+00$ & $0.000000e+00$ & $0.000000e+00$ & $0.000000e+00$ \\
& $-nan$ & $-nan$ & $-nan$ & $-nan$ \\
& $-nan$ & $-nan$ & $-nan$ & $-nan$ \\
& $-nan$ & $-nan$ & $-nan$ & $-nan$ \\
\hline
$1.000000e-02$ & $0.000000e+00$ & $0.000000e+00$ & $0.000000e+00$ & $0.000000e+00$ \\
& $-nan$ & $-nan$ & $-nan$ & $-nan$ \\
& $-nan$ & $-nan$ & $-nan$ & $-nan$ \\
& $-nan$ & $-nan$ & $-nan$ & $-nan$ \\
\hline
$1.000000e-03$ & $0.000000e+00$ & $3.769582e-03$ & $0.000000e+00$ & $0.000000e+00$ \\
& $-nan$ & $1.298738e-03$ & $-nan$ & $-nan$ \\
& $-nan$ & $1.590623e-03$ & $-nan$ & $-nan$ \\
& $-nan$ & $2.018589e-03$ & $-nan$ & $-nan$ \\
\hline
$1.000000e-04$ & $0.000000e+00$ & $3.675397e-03$ & $4.743583e-05$ & $9.941056e-06$ \\
& $-nan$ & $1.278853e-03$ & $1.581963e-05$ & $6.074528e-06$ \\
& $-nan$ & $1.566269e-03$ & $1.937501e-05$ & $7.439747e-06$ \\
& $-nan$ & $1.988361e-03$ & $2.496801e-05$ & $9.604346e-06$ \\
\hline
\end{tabular}


\begin{tabular}{ |l|l|l|l|l| }
\hline
\multicolumn{5}{|c|}{$\mu = 1.000000e-03, p(\rho) = 1.000000e+00\rho^{1.4}$} \\
\hline
$\tau\setminus h$ & $0.1$ & $0.01$ & $0.001$ & $0.0001$\\
\hline
$1.000000e-01$ & $1.352222e+01$ & $0.000000e+00$ & $0.000000e+00$ & $0.000000e+00$ \\
& $7.779070e+00$ & $-nan$ & $-nan$ & $-nan$ \\
& $9.527377e+00$ & $-nan$ & $-nan$ & $-nan$ \\
& $1.214943e+01$ & $-nan$ & $-nan$ & $-nan$ \\
\hline
$1.000000e-02$ & $0.000000e+00$ & $0.000000e+00$ & $0.000000e+00$ & $0.000000e+00$ \\
& $-nan$ & $-nan$ & $-nan$ & $-nan$ \\
& $-nan$ & $-nan$ & $-nan$ & $-nan$ \\
& $-nan$ & $-nan$ & $-nan$ & $-nan$ \\
\hline
$1.000000e-03$ & $0.000000e+00$ & $2.042115e-02$ & $9.710140e-03$ & $0.000000e+00$ \\
& $-nan$ & $6.633283e-03$ & $4.745163e-03$ & $-nan$ \\
& $-nan$ & $8.124079e-03$ & $5.811614e-03$ & $-nan$ \\
& $-nan$ & $1.029084e-02$ & $7.496476e-03$ & $-nan$ \\
\hline
$1.000000e-04$ & $0.000000e+00$ & $1.180180e-02$ & $1.046818e-03$ & $9.576699e-04$ \\
& $-nan$ & $3.303458e-03$ & $4.819182e-04$ & $4.706194e-04$ \\
& $-nan$ & $4.045894e-03$ & $5.902268e-04$ & $5.763886e-04$ \\
& $-nan$ & $5.089795e-03$ & $7.612602e-04$ & $7.440529e-04$ \\
\hline
\end{tabular}


\begin{tabular}{ |l|l|l|l|l| }
\hline
\multicolumn{5}{|c|}{$\mu = 1.000000e-03, p(\rho) = 1.000000e+00\rho$} \\
\hline
$\tau\setminus h$ & $0.1$ & $0.01$ & $0.001$ & $0.0001$\\
\hline
$1.000000e-01$ & $2.660592e+00$ & $0.000000e+00$ & $4.100188e+02$ & $0.000000e+00$ \\
& $1.466579e+00$ & $-nan$ & $2.105526e+01$ & $-nan$ \\
& $1.796186e+00$ & $-nan$ & $2.578732e+01$ & $-nan$ \\
& $2.295273e+00$ & $-nan$ & $3.066258e+01$ & $-nan$ \\
\hline
$1.000000e-02$ & $0.000000e+00$ & $2.367952e-01$ & $0.000000e+00$ & $0.000000e+00$ \\
& $-nan$ & $8.698023e-02$ & $-nan$ & $-nan$ \\
& $-nan$ & $1.065286e-01$ & $-nan$ & $-nan$ \\
& $-nan$ & $1.372941e-01$ & $-nan$ & $-nan$ \\
\hline
$1.000000e-03$ & $0.000000e+00$ & $2.076761e-02$ & $2.480389e-02$ & $2.488168e-02$ \\
& $-nan$ & $6.880710e-03$ & $8.677294e-03$ & $8.699589e-03$ \\
& $-nan$ & $8.427114e-03$ & $1.062747e-02$ & $1.065478e-02$ \\
& $-nan$ & $1.082729e-02$ & $1.371750e-02$ & $1.375501e-02$ \\
\hline
$1.000000e-04$ & $0.000000e+00$ & $6.665540e-03$ & $2.404710e-03$ & $2.421496e-03$ \\
& $-nan$ & $3.556954e-03$ & $8.450839e-04$ & $8.705230e-04$ \\
& $-nan$ & $4.356362e-03$ & $1.035012e-03$ & $1.066169e-03$ \\
& $-nan$ & $5.615356e-03$ & $1.335919e-03$ & $1.376392e-03$ \\
\hline
\end{tabular}


\begin{tabular}{ |l|l|l|l|l| }
\hline
\multicolumn{5}{|c|}{$\mu = 1.000000e-03, p(\rho) = 1.000000e+01\rho^{1.4}$} \\
\hline
$\tau\setminus h$ & $0.1$ & $0.01$ & $0.001$ & $0.0001$\\
\hline
$1.000000e-01$ & $1.352222e+01$ & $0.000000e+00$ & $0.000000e+00$ & $0.000000e+00$ \\
& $7.779070e+00$ & $-nan$ & $-nan$ & $-nan$ \\
& $9.527377e+00$ & $-nan$ & $-nan$ & $-nan$ \\
& $1.214943e+01$ & $-nan$ & $-nan$ & $-nan$ \\
\hline
$1.000000e-02$ & $0.000000e+00$ & $0.000000e+00$ & $0.000000e+00$ & $0.000000e+00$ \\
& $-nan$ & $-nan$ & $-nan$ & $-nan$ \\
& $-nan$ & $-nan$ & $-nan$ & $-nan$ \\
& $-nan$ & $-nan$ & $-nan$ & $-nan$ \\
\hline
$1.000000e-03$ & $0.000000e+00$ & $2.042115e-02$ & $9.710140e-03$ & $0.000000e+00$ \\
& $-nan$ & $6.633283e-03$ & $4.745163e-03$ & $-nan$ \\
& $-nan$ & $8.124079e-03$ & $5.811614e-03$ & $-nan$ \\
& $-nan$ & $1.029084e-02$ & $7.496476e-03$ & $-nan$ \\
\hline
$1.000000e-04$ & $0.000000e+00$ & $1.180180e-02$ & $1.046818e-03$ & $9.576699e-04$ \\
& $-nan$ & $3.303458e-03$ & $4.819182e-04$ & $4.706194e-04$ \\
& $-nan$ & $4.045894e-03$ & $5.902268e-04$ & $5.763886e-04$ \\
& $-nan$ & $5.089795e-03$ & $7.612602e-04$ & $7.440529e-04$ \\
\hline
\end{tabular}


\begin{tabular}{ |l|l|l|l|l| }
\hline
\multicolumn{5}{|c|}{$\mu = 1.000000e-03, p(\rho) = 1.000000e+01\rho$} \\
\hline
$\tau\setminus h$ & $0.1$ & $0.01$ & $0.001$ & $0.0001$\\
\hline
$1.000000e-01$ & $0.000000e+00$ & $0.000000e+00$ & $0.000000e+00$ & $0.000000e+00$ \\
& $-nan$ & $-nan$ & $-nan$ & $-nan$ \\
& $-nan$ & $-nan$ & $-nan$ & $-nan$ \\
& $-nan$ & $-nan$ & $-nan$ & $-nan$ \\
\hline
$1.000000e-02$ & $0.000000e+00$ & $0.000000e+00$ & $0.000000e+00$ & $0.000000e+00$ \\
& $-nan$ & $-nan$ & $-nan$ & $-nan$ \\
& $-nan$ & $-nan$ & $-nan$ & $-nan$ \\
& $-nan$ & $-nan$ & $-nan$ & $-nan$ \\
\hline
$1.000000e-03$ & $0.000000e+00$ & $3.888350e-03$ & $0.000000e+00$ & $0.000000e+00$ \\
& $-nan$ & $1.923126e-03$ & $-nan$ & $-nan$ \\
& $-nan$ & $2.355339e-03$ & $-nan$ & $-nan$ \\
& $-nan$ & $3.017394e-03$ & $-nan$ & $-nan$ \\
\hline
$1.000000e-04$ & $0.000000e+00$ & $4.649429e-03$ & $1.388821e-04$ & $1.444728e-04$ \\
& $-nan$ & $2.160396e-03$ & $7.577885e-05$ & $8.562781e-05$ \\
& $-nan$ & $2.645934e-03$ & $9.280975e-05$ & $1.048722e-04$ \\
& $-nan$ & $3.387362e-03$ & $1.197991e-04$ & $1.353858e-04$ \\
\hline
\end{tabular}


\begin{tabular}{ |l|l|l|l|l| }
\hline
\multicolumn{5}{|c|}{$\mu = 1.000000e-03, p(\rho) = 1.000000e+02\rho^{1.4}$} \\
\hline
$\tau\setminus h$ & $0.1$ & $0.01$ & $0.001$ & $0.0001$\\
\hline
$1.000000e-01$ & $1.352222e+01$ & $0.000000e+00$ & $0.000000e+00$ & $0.000000e+00$ \\
& $7.779070e+00$ & $-nan$ & $-nan$ & $-nan$ \\
& $9.527377e+00$ & $-nan$ & $-nan$ & $-nan$ \\
& $1.214943e+01$ & $-nan$ & $-nan$ & $-nan$ \\
\hline
$1.000000e-02$ & $0.000000e+00$ & $0.000000e+00$ & $0.000000e+00$ & $0.000000e+00$ \\
& $-nan$ & $-nan$ & $-nan$ & $-nan$ \\
& $-nan$ & $-nan$ & $-nan$ & $-nan$ \\
& $-nan$ & $-nan$ & $-nan$ & $-nan$ \\
\hline
$1.000000e-03$ & $0.000000e+00$ & $2.042115e-02$ & $9.710140e-03$ & $0.000000e+00$ \\
& $-nan$ & $6.633283e-03$ & $4.745163e-03$ & $-nan$ \\
& $-nan$ & $8.124079e-03$ & $5.811614e-03$ & $-nan$ \\
& $-nan$ & $1.029084e-02$ & $7.496476e-03$ & $-nan$ \\
\hline
$1.000000e-04$ & $0.000000e+00$ & $1.180180e-02$ & $1.046818e-03$ & $9.576699e-04$ \\
& $-nan$ & $3.303458e-03$ & $4.819182e-04$ & $4.706194e-04$ \\
& $-nan$ & $4.045894e-03$ & $5.902268e-04$ & $5.763886e-04$ \\
& $-nan$ & $5.089795e-03$ & $7.612602e-04$ & $7.440529e-04$ \\
\hline
\end{tabular}


\begin{tabular}{ |l|l|l|l|l| }
\hline
\multicolumn{5}{|c|}{$\mu = 1.000000e-03, p(\rho) = 1.000000e+02\rho$} \\
\hline
$\tau\setminus h$ & $0.1$ & $0.01$ & $0.001$ & $0.0001$\\
\hline
$1.000000e-01$ & $0.000000e+00$ & $0.000000e+00$ & $0.000000e+00$ & $0.000000e+00$ \\
& $-nan$ & $-nan$ & $-nan$ & $-nan$ \\
& $-nan$ & $-nan$ & $-nan$ & $-nan$ \\
& $-nan$ & $-nan$ & $-nan$ & $-nan$ \\
\hline
$1.000000e-02$ & $0.000000e+00$ & $0.000000e+00$ & $0.000000e+00$ & $0.000000e+00$ \\
& $-nan$ & $-nan$ & $-nan$ & $-nan$ \\
& $-nan$ & $-nan$ & $-nan$ & $-nan$ \\
& $-nan$ & $-nan$ & $-nan$ & $-nan$ \\
\hline
$1.000000e-03$ & $0.000000e+00$ & $0.000000e+00$ & $0.000000e+00$ & $0.000000e+00$ \\
& $-nan$ & $-nan$ & $-nan$ & $-nan$ \\
& $-nan$ & $-nan$ & $-nan$ & $-nan$ \\
& $-nan$ & $-nan$ & $-nan$ & $-nan$ \\
\hline
$1.000000e-04$ & $0.000000e+00$ & $4.584840e-03$ & $4.969368e-05$ & $0.000000e+00$ \\
& $-nan$ & $1.420135e-03$ & $1.708520e-05$ & $-nan$ \\
& $-nan$ & $1.739303e-03$ & $2.092501e-05$ & $-nan$ \\
& $-nan$ & $2.198124e-03$ & $2.696833e-05$ & $-nan$ \\
\hline
\end{tabular}




% Для тестового запуска используйте:
% \begin{tabular}{ |l|l|l|l|l| }
\hline
\multicolumn{5}{|c|}{$\mu = 0.1, p(\rho) = 1\rho$} \\
\hline
$\tau\setminus h$ & $0.1$ & $0.01$ & $0.001$ & $0.0001$\\
\hline
$0.1$ & $9.417089e+00$ & $7.134958e+00$ \\
& $5.545571e+00$ & $2.299721e+00$ \\
& $3.847637e+01$ & $6.532080e+01$ \\
& $1.000000e-02$ & $1.000000e-03$ \\
\hline
$1.000000e-02$ & $2.922101e+02$ & $1.954389e-01$ \\
& $1.094415e+02$ & $9.356493e-02$ \\
& $7.356948e+02$ & $1.238827e+00$ \\
& $1.000000e-03$ & $1.000000e-04$ \\
\hline
\end{tabular}


\begin{tabular}{ |l|l|l|l|l| }
\hline
\multicolumn{5}{|c|}{$\mu = 1.000000e-01, p(\rho) = 1.000000e+01\rho$} \\
\hline
$\tau\setminus h$ & $0.1$ & $0.01$ & $0.001$ & $0.0001$\\
\hline
$1.000000e-01$ & $3.331907e+03$ & $2.238448e+01$ \\
& $1.627620e+03$ & $5.610216e+00$ \\
& $1.079715e+04$ & $1.097938e+02$ \\
& $1.000000e-02$ & $1.000000e-03$ \\
\hline
$1.000000e-02$ & $6.143129e+01$ & $1.009663e-02$ \\
& $2.752249e+01$ & $7.239306e-03$ \\
& $1.117193e+02$ & $2.520551e-02$ \\
& $1.000000e-03$ & $1.000000e-04$ \\
\hline
\end{tabular}




\section{Результаты для V (скорость)}

% Вставка таблиц для V
% Раскомментируйте следующую строку после генерации таблиц:
% \begin{tabular}{ |l|l|l|l|l| }
\hline
\multicolumn{5}{|c|}{$\mu = 0.1, p(\rho) = 1\rho^{1.4}$} \\
\hline
$\tau\setminus h$ & $0.1$ & $0.01$ & $0.001$ & $0.0001$\\
\hline
$0.1$ & $1.069504e+00$ & $2.581995e+18$ & $0.000000e+00$ & $0.000000e+00$ \\
& $6.136821e-01$ & $6.202478e+17$ & $-nan$ & $-nan$ \\
& $7.516041e-01$ & $7.596453e+17$ & $-nan$ & $-nan$ \\
& $9.703167e-01$ & $9.806979e+17$ & $-nan$ & $-nan$ \\
\hline
$1.000000e-02$ & $6.639447e-01$ & $3.603866e-02$ & $3.340054e-02$ & $3.337361e-02$ \\
& $3.607850e-01$ & $1.992096e-02$ & $1.739419e-02$ & $1.736992e-02$ \\
& $4.418695e-01$ & $2.439809e-02$ & $2.130345e-02$ & $2.127372e-02$ \\
& $5.704511e-01$ & $3.149780e-02$ & $2.750263e-02$ & $2.746425e-02$ \\
\hline
$1.000000e-03$ & $2.240642e+00$ & $8.396779e-03$ & $3.061933e-03$ & $3.009031e-03$ \\
& $1.350211e+00$ & $4.479272e-03$ & $1.748288e-03$ & $1.722470e-03$ \\
& $1.653664e+00$ & $5.485965e-03$ & $2.141207e-03$ & $2.109587e-03$ \\
& $2.134871e+00$ & $7.082351e-03$ & $2.764287e-03$ & $2.723465e-03$ \\
\hline
$1.000000e-04$ & $2.316121e+00$ & $5.728309e-03$ & $3.607471e-04$ & $3.075881e-04$ \\
& $1.375152e+00$ & $3.011930e-03$ & $1.985557e-04$ & $1.723888e-04$ \\
& $1.684210e+00$ & $3.688846e-03$ & $2.431801e-04$ & $2.111323e-04$ \\
& $2.174306e+00$ & $4.762279e-03$ & $3.139442e-04$ & $2.725707e-04$ \\
\hline
\end{tabular}


\begin{tabular}{ |l|l|l|l|l| }
\hline
\multicolumn{5}{|c|}{$\mu = 1.000000e-01, p(\rho) = 1.000000e+00\rho$} \\
\hline
$\tau\setminus h$ & $0.1$ & $0.01$ & $0.001$ & $0.0001$\\
\hline
$1.000000e-01$ & $1.119882e+00$ & $1.625199e+00$ & $1.512012e+00$ & $1.504856e+00$ \\
& $6.826320e-01$ & $7.242010e-01$ & $6.962480e-01$ & $6.940562e-01$ \\
& $8.360501e-01$ & $8.869615e-01$ & $8.527261e-01$ & $8.500418e-01$ \\
& $1.079336e+00$ & $1.145062e+00$ & $1.100865e+00$ & $1.097399e+00$ \\
\hline
$1.000000e-02$ & $3.505647e-01$ & $1.675353e-01$ & $1.617196e-01$ & $1.616627e-01$ \\
& $1.862471e-01$ & $7.960856e-02$ & $7.619895e-02$ & $7.616304e-02$ \\
& $2.281052e-01$ & $9.750018e-02$ & $9.332427e-02$ & $9.328029e-02$ \\
& $2.944825e-01$ & $1.258722e-01$ & $1.204811e-01$ & $1.204243e-01$ \\
\hline
$1.000000e-03$ & $3.547002e+00$ & $2.106987e-02$ & $1.599773e-02$ & $1.594741e-02$ \\
& $1.429314e+00$ & $1.010633e-02$ & $7.097150e-03$ & $7.068355e-03$ \\
& $1.750546e+00$ & $1.237768e-02$ & $8.692198e-03$ & $8.656931e-03$ \\
& $2.259945e+00$ & $1.597951e-02$ & $1.122158e-02$ & $1.117605e-02$ \\
\hline
$1.000000e-04$ & $2.986653e+00$ & $6.759643e-03$ & $1.641338e-03$ & $1.591866e-03$ \\
& $1.395251e+00$ & $3.866873e-03$ & $7.296485e-04$ & $7.012338e-04$ \\
& $1.708826e+00$ & $4.735933e-03$ & $8.936333e-04$ & $8.588326e-04$ \\
& $2.206085e+00$ & $6.114064e-03$ & $1.153676e-03$ & $1.108748e-03$ \\
\hline
\end{tabular}


\begin{tabular}{ |l|l|l|l|l| }
\hline
\multicolumn{5}{|c|}{$\mu = 1.000000e-01, p(\rho) = 1.000000e+01\rho^{1.4}$} \\
\hline
$\tau\setminus h$ & $0.1$ & $0.01$ & $0.001$ & $0.0001$\\
\hline
$1.000000e-01$ & $1.069504e+00$ & $2.581995e+18$ & $0.000000e+00$ & $0.000000e+00$ \\
& $6.136821e-01$ & $6.202478e+17$ & $-nan$ & $-nan$ \\
& $7.516041e-01$ & $7.596453e+17$ & $-nan$ & $-nan$ \\
& $9.703167e-01$ & $9.806979e+17$ & $-nan$ & $-nan$ \\
\hline
$1.000000e-02$ & $6.639447e-01$ & $3.603866e-02$ & $3.340054e-02$ & $3.337361e-02$ \\
& $3.607850e-01$ & $1.992096e-02$ & $1.739419e-02$ & $1.736992e-02$ \\
& $4.418695e-01$ & $2.439809e-02$ & $2.130345e-02$ & $2.127372e-02$ \\
& $5.704511e-01$ & $3.149780e-02$ & $2.750263e-02$ & $2.746425e-02$ \\
\hline
$1.000000e-03$ & $2.240642e+00$ & $8.396779e-03$ & $3.061933e-03$ & $3.009031e-03$ \\
& $1.350211e+00$ & $4.479272e-03$ & $1.748288e-03$ & $1.722470e-03$ \\
& $1.653664e+00$ & $5.485965e-03$ & $2.141207e-03$ & $2.109587e-03$ \\
& $2.134871e+00$ & $7.082351e-03$ & $2.764287e-03$ & $2.723465e-03$ \\
\hline
$1.000000e-04$ & $2.316121e+00$ & $5.728309e-03$ & $3.607471e-04$ & $3.075881e-04$ \\
& $1.375152e+00$ & $3.011930e-03$ & $1.985557e-04$ & $1.723888e-04$ \\
& $1.684210e+00$ & $3.688846e-03$ & $2.431801e-04$ & $2.111323e-04$ \\
& $2.174306e+00$ & $4.762279e-03$ & $3.139442e-04$ & $2.725707e-04$ \\
\hline
\end{tabular}


\begin{tabular}{ |l|l|l|l|l| }
\hline
\multicolumn{5}{|c|}{$\mu = 1.000000e-01, p(\rho) = 1.000000e+01\rho$} \\
\hline
$\tau\setminus h$ & $0.1$ & $0.01$ & $0.001$ & $0.0001$\\
\hline
$1.000000e-01$ & $0.000000e+00$ & $3.288957e+00$ & $0.000000e+00$ & $2.263396e+01$ \\
& $-nan$ & $1.781050e+00$ & $-nan$ & $1.038287e+01$ \\
& $-nan$ & $2.181331e+00$ & $-nan$ & $1.271637e+01$ \\
& $-nan$ & $2.816087e+00$ & $-nan$ & $1.641676e+01$ \\
\hline
$1.000000e-02$ & $2.599785e+01$ & $1.802300e-02$ & $1.429668e-02$ & $1.425872e-02$ \\
& $2.025212e+01$ & $6.928088e-03$ & $5.626667e-03$ & $5.619529e-03$ \\
& $2.480368e+01$ & $8.485141e-03$ & $6.891232e-03$ & $6.882489e-03$ \\
& $3.202142e+01$ & $1.095427e-02$ & $8.896542e-03$ & $8.885255e-03$ \\
\hline
$1.000000e-03$ & $1.035950e+02$ & $5.667965e-03$ & $1.287664e-03$ & $1.252654e-03$ \\
& $6.712537e+01$ & $3.088514e-03$ & $4.845470e-04$ & $4.727778e-04$ \\
& $8.221145e+01$ & $3.782642e-03$ & $5.934465e-04$ & $5.790322e-04$ \\
& $1.061345e+02$ & $4.883370e-03$ & $7.661361e-04$ & $7.475273e-04$ \\
\hline
$1.000000e-04$ & $0.000000e+00$ & $5.585547e-03$ & $1.590194e-04$ & $1.241327e-04$ \\
& $-nan$ & $2.883446e-03$ & $6.391986e-05$ & $4.674609e-05$ \\
& $-nan$ & $3.531486e-03$ & $7.828552e-05$ & $5.725204e-05$ \\
& $-nan$ & $4.559128e-03$ & $1.010662e-04$ & $7.391206e-05$ \\
\hline
\end{tabular}


\begin{tabular}{ |l|l|l|l|l| }
\hline
\multicolumn{5}{|c|}{$\mu = 1.000000e-01, p(\rho) = 1.000000e+02\rho^{1.4}$} \\
\hline
$\tau\setminus h$ & $0.1$ & $0.01$ & $0.001$ & $0.0001$\\
\hline
$1.000000e-01$ & $1.069504e+00$ & $2.581995e+18$ & $0.000000e+00$ & $0.000000e+00$ \\
& $6.136821e-01$ & $6.202478e+17$ & $-nan$ & $-nan$ \\
& $7.516041e-01$ & $7.596453e+17$ & $-nan$ & $-nan$ \\
& $9.703167e-01$ & $9.806979e+17$ & $-nan$ & $-nan$ \\
\hline
$1.000000e-02$ & $6.639447e-01$ & $3.603866e-02$ & $3.340054e-02$ & $3.337361e-02$ \\
& $3.607850e-01$ & $1.992096e-02$ & $1.739419e-02$ & $1.736992e-02$ \\
& $4.418695e-01$ & $2.439809e-02$ & $2.130345e-02$ & $2.127372e-02$ \\
& $5.704511e-01$ & $3.149780e-02$ & $2.750263e-02$ & $2.746425e-02$ \\
\hline
$1.000000e-03$ & $2.240642e+00$ & $8.396779e-03$ & $3.061933e-03$ & $3.009031e-03$ \\
& $1.350211e+00$ & $4.479272e-03$ & $1.748288e-03$ & $1.722470e-03$ \\
& $1.653664e+00$ & $5.485965e-03$ & $2.141207e-03$ & $2.109587e-03$ \\
& $2.134871e+00$ & $7.082351e-03$ & $2.764287e-03$ & $2.723465e-03$ \\
\hline
$1.000000e-04$ & $2.316121e+00$ & $5.728309e-03$ & $3.607471e-04$ & $3.075881e-04$ \\
& $1.375152e+00$ & $3.011930e-03$ & $1.985557e-04$ & $1.723888e-04$ \\
& $1.684210e+00$ & $3.688846e-03$ & $2.431801e-04$ & $2.111323e-04$ \\
& $2.174306e+00$ & $4.762279e-03$ & $3.139442e-04$ & $2.725707e-04$ \\
\hline
\end{tabular}


\begin{tabular}{ |l|l|l|l|l| }
\hline
\multicolumn{5}{|c|}{$\mu = 1.000000e-01, p(\rho) = 1.000000e+02\rho$} \\
\hline
$\tau\setminus h$ & $0.1$ & $0.01$ & $0.001$ & $0.0001$\\
\hline
$1.000000e-01$ & $0.000000e+00$ & $0.000000e+00$ & $0.000000e+00$ & $0.000000e+00$ \\
& $-nan$ & $-nan$ & $-nan$ & $-nan$ \\
& $-nan$ & $-nan$ & $-nan$ & $-nan$ \\
& $-nan$ & $-nan$ & $-nan$ & $-nan$ \\
\hline
$1.000000e-02$ & $0.000000e+00$ & $0.000000e+00$ & $0.000000e+00$ & $0.000000e+00$ \\
& $-nan$ & $-nan$ & $-nan$ & $-nan$ \\
& $-nan$ & $-nan$ & $-nan$ & $-nan$ \\
& $-nan$ & $-nan$ & $-nan$ & $-nan$ \\
\hline
$1.000000e-03$ & $0.000000e+00$ & $3.522692e-03$ & $7.066974e-04$ & $7.477169e-04$ \\
& $-nan$ & $1.928307e-03$ & $4.767697e-04$ & $4.899937e-04$ \\
& $-nan$ & $2.361684e-03$ & $5.839212e-04$ & $6.001173e-04$ \\
& $-nan$ & $3.048921e-03$ & $7.538391e-04$ & $7.747481e-04$ \\
\hline
$1.000000e-04$ & $0.000000e+00$ & $4.210602e-03$ & $7.384266e-05$ & $7.282831e-05$ \\
& $-nan$ & $2.186824e-03$ & $3.936274e-05$ & $4.845407e-05$ \\
& $-nan$ & $2.678302e-03$ & $4.820932e-05$ & $5.934387e-05$ \\
& $-nan$ & $3.457673e-03$ & $6.223796e-05$ & $7.661261e-05$ \\
\hline
\end{tabular}


\begin{tabular}{ |l|l|l|l|l| }
\hline
\multicolumn{5}{|c|}{$\mu = 1.000000e-02, p(\rho) = 1.000000e+00\rho^{1.4}$} \\
\hline
$\tau\setminus h$ & $0.1$ & $0.01$ & $0.001$ & $0.0001$\\
\hline
$1.000000e-01$ & $5.507609e+01$ & $0.000000e+00$ & $0.000000e+00$ & $0.000000e+00$ \\
& $2.398661e+01$ & $-nan$ & $-nan$ & $-nan$ \\
& $2.937748e+01$ & $-nan$ & $-nan$ & $-nan$ \\
& $3.792616e+01$ & $-nan$ & $-nan$ & $-nan$ \\
\hline
$1.000000e-02$ & $0.000000e+00$ & $6.184536e-02$ & $5.318321e-02$ & $5.309446e-02$ \\
& $-nan$ & $2.158002e-02$ & $1.881303e-02$ & $1.879874e-02$ \\
& $-nan$ & $2.643001e-02$ & $2.304117e-02$ & $2.302366e-02$ \\
& $-nan$ & $3.412100e-02$ & $2.974602e-02$ & $2.972341e-02$ \\
\hline
$1.000000e-03$ & $0.000000e+00$ & $1.151356e-02$ & $4.275408e-03$ & $4.207118e-03$ \\
& $-nan$ & $4.758089e-03$ & $1.706533e-03$ & $1.684025e-03$ \\
& $-nan$ & $5.827446e-03$ & $2.090067e-03$ & $2.062501e-03$ \\
& $-nan$ & $7.523200e-03$ & $2.698265e-03$ & $2.662678e-03$ \\
\hline
$1.000000e-04$ & $0.000000e+00$ & $8.285609e-03$ & $4.793451e-04$ & $4.120143e-04$ \\
& $-nan$ & $3.596280e-03$ & $1.907603e-04$ & $1.667531e-04$ \\
& $-nan$ & $4.404525e-03$ & $2.336327e-04$ & $2.042299e-04$ \\
& $-nan$ & $5.686217e-03$ & $3.016185e-04$ & $2.636597e-04$ \\
\hline
\end{tabular}


\begin{tabular}{ |l|l|l|l|l| }
\hline
\multicolumn{5}{|c|}{$\mu = 1.000000e-02, p(\rho) = 1.000000e+00\rho$} \\
\hline
$\tau\setminus h$ & $0.1$ & $0.01$ & $0.001$ & $0.0001$\\
\hline
$1.000000e-01$ & $4.332669e+00$ & $1.225367e+02$ & $2.231718e+01$ & $3.673132e+00$ \\
& $2.939219e+00$ & $5.682859e+01$ & $1.268277e+01$ & $1.747116e+00$ \\
& $3.599794e+00$ & $6.960052e+01$ & $1.553316e+01$ & $2.139771e+00$ \\
& $4.647314e+00$ & $8.985388e+01$ & $2.005322e+01$ & $2.762433e+00$ \\
\hline
$1.000000e-02$ & $2.560329e+01$ & $1.003913e-01$ & $9.845647e-02$ & $9.844481e-02$ \\
& $1.185275e+01$ & $3.827838e-02$ & $3.596100e-02$ & $3.593865e-02$ \\
& $1.451659e+01$ & $4.688125e-02$ & $4.404305e-02$ & $4.401568e-02$ \\
& $1.874084e+01$ & $6.052343e-02$ & $5.685933e-02$ & $5.682400e-02$ \\
\hline
$1.000000e-03$ & $0.000000e+00$ & $1.408349e-02$ & $9.316930e-03$ & $9.273767e-03$ \\
& $-nan$ & $6.355314e-03$ & $3.738807e-03$ & $3.717382e-03$ \\
& $-nan$ & $7.783638e-03$ & $4.579085e-03$ & $4.552844e-03$ \\
& $-nan$ & $1.004863e-02$ & $5.911573e-03$ & $5.877696e-03$ \\
\hline
$1.000000e-04$ & $0.000000e+00$ & $7.967103e-03$ & $9.656950e-04$ & $9.198125e-04$ \\
& $-nan$ & $3.494248e-03$ & $3.950044e-04$ & $3.728564e-04$ \\
& $-nan$ & $4.279563e-03$ & $4.837796e-04$ & $4.566539e-04$ \\
& $-nan$ & $5.524892e-03$ & $6.245568e-04$ & $5.895377e-04$ \\
\hline
\end{tabular}


\begin{tabular}{ |l|l|l|l|l| }
\hline
\multicolumn{5}{|c|}{$\mu = 1.000000e-02, p(\rho) = 1.000000e+01\rho^{1.4}$} \\
\hline
$\tau\setminus h$ & $0.1$ & $0.01$ & $0.001$ & $0.0001$\\
\hline
$1.000000e-01$ & $5.507609e+01$ & $0.000000e+00$ & $0.000000e+00$ & $0.000000e+00$ \\
& $2.398661e+01$ & $-nan$ & $-nan$ & $-nan$ \\
& $2.937748e+01$ & $-nan$ & $-nan$ & $-nan$ \\
& $3.792616e+01$ & $-nan$ & $-nan$ & $-nan$ \\
\hline
$1.000000e-02$ & $0.000000e+00$ & $6.184536e-02$ & $5.318321e-02$ & $5.309446e-02$ \\
& $-nan$ & $2.158002e-02$ & $1.881303e-02$ & $1.879874e-02$ \\
& $-nan$ & $2.643001e-02$ & $2.304117e-02$ & $2.302366e-02$ \\
& $-nan$ & $3.412100e-02$ & $2.974602e-02$ & $2.972341e-02$ \\
\hline
$1.000000e-03$ & $0.000000e+00$ & $1.151356e-02$ & $4.275408e-03$ & $4.207118e-03$ \\
& $-nan$ & $4.758089e-03$ & $1.706533e-03$ & $1.684025e-03$ \\
& $-nan$ & $5.827446e-03$ & $2.090067e-03$ & $2.062501e-03$ \\
& $-nan$ & $7.523200e-03$ & $2.698265e-03$ & $2.662678e-03$ \\
\hline
$1.000000e-04$ & $0.000000e+00$ & $8.285609e-03$ & $4.793451e-04$ & $4.120143e-04$ \\
& $-nan$ & $3.596280e-03$ & $1.907603e-04$ & $1.667531e-04$ \\
& $-nan$ & $4.404525e-03$ & $2.336327e-04$ & $2.042299e-04$ \\
& $-nan$ & $5.686217e-03$ & $3.016185e-04$ & $2.636597e-04$ \\
\hline
\end{tabular}


\begin{tabular}{ |l|l|l|l|l| }
\hline
\multicolumn{5}{|c|}{$\mu = 1.000000e-02, p(\rho) = 1.000000e+01\rho$} \\
\hline
$\tau\setminus h$ & $0.1$ & $0.01$ & $0.001$ & $0.0001$\\
\hline
$1.000000e-01$ & $0.000000e+00$ & $0.000000e+00$ & $1.125698e+01$ & $9.444721e+00$ \\
& $-nan$ & $-nan$ & $4.105701e+00$ & $3.889572e+00$ \\
& $-nan$ & $-nan$ & $5.028436e+00$ & $4.763734e+00$ \\
& $-nan$ & $-nan$ & $6.491684e+00$ & $6.149954e+00$ \\
\hline
$1.000000e-02$ & $0.000000e+00$ & $0.000000e+00$ & $0.000000e+00$ & $0.000000e+00$ \\
& $-nan$ & $-nan$ & $-nan$ & $-nan$ \\
& $-nan$ & $-nan$ & $-nan$ & $-nan$ \\
& $-nan$ & $-nan$ & $-nan$ & $-nan$ \\
\hline
$1.000000e-03$ & $0.000000e+00$ & $9.701823e-03$ & $2.331505e-03$ & $2.324408e-03$ \\
& $-nan$ & $4.117890e-03$ & $1.134013e-03$ & $1.150422e-03$ \\
& $-nan$ & $5.043365e-03$ & $1.388877e-03$ & $1.408974e-03$ \\
& $-nan$ & $6.510957e-03$ & $1.793032e-03$ & $1.818978e-03$ \\
\hline
$1.000000e-04$ & $0.000000e+00$ & $1.054342e-02$ & $2.387926e-04$ & $2.302748e-04$ \\
& $-nan$ & $4.347372e-03$ & $1.051778e-04$ & $1.134323e-04$ \\
& $-nan$ & $5.324421e-03$ & $1.288159e-04$ & $1.389256e-04$ \\
& $-nan$ & $6.873798e-03$ & $1.663007e-04$ & $1.793521e-04$ \\
\hline
\end{tabular}


\begin{tabular}{ |l|l|l|l|l| }
\hline
\multicolumn{5}{|c|}{$\mu = 1.000000e-02, p(\rho) = 1.000000e+02\rho^{1.4}$} \\
\hline
$\tau\setminus h$ & $0.1$ & $0.01$ & $0.001$ & $0.0001$\\
\hline
$1.000000e-01$ & $5.507609e+01$ & $0.000000e+00$ & $0.000000e+00$ & $0.000000e+00$ \\
& $2.398661e+01$ & $-nan$ & $-nan$ & $-nan$ \\
& $2.937748e+01$ & $-nan$ & $-nan$ & $-nan$ \\
& $3.792616e+01$ & $-nan$ & $-nan$ & $-nan$ \\
\hline
$1.000000e-02$ & $0.000000e+00$ & $6.184536e-02$ & $5.318321e-02$ & $5.309446e-02$ \\
& $-nan$ & $2.158002e-02$ & $1.881303e-02$ & $1.879874e-02$ \\
& $-nan$ & $2.643001e-02$ & $2.304117e-02$ & $2.302366e-02$ \\
& $-nan$ & $3.412100e-02$ & $2.974602e-02$ & $2.972341e-02$ \\
\hline
$1.000000e-03$ & $0.000000e+00$ & $1.151356e-02$ & $4.275408e-03$ & $4.207118e-03$ \\
& $-nan$ & $4.758089e-03$ & $1.706533e-03$ & $1.684025e-03$ \\
& $-nan$ & $5.827446e-03$ & $2.090067e-03$ & $2.062501e-03$ \\
& $-nan$ & $7.523200e-03$ & $2.698265e-03$ & $2.662678e-03$ \\
\hline
$1.000000e-04$ & $0.000000e+00$ & $8.285609e-03$ & $4.793451e-04$ & $4.120143e-04$ \\
& $-nan$ & $3.596280e-03$ & $1.907603e-04$ & $1.667531e-04$ \\
& $-nan$ & $4.404525e-03$ & $2.336327e-04$ & $2.042299e-04$ \\
& $-nan$ & $5.686217e-03$ & $3.016185e-04$ & $2.636597e-04$ \\
\hline
\end{tabular}


\begin{tabular}{ |l|l|l|l|l| }
\hline
\multicolumn{5}{|c|}{$\mu = 1.000000e-02, p(\rho) = 1.000000e+02\rho$} \\
\hline
$\tau\setminus h$ & $0.1$ & $0.01$ & $0.001$ & $0.0001$\\
\hline
$1.000000e-01$ & $0.000000e+00$ & $0.000000e+00$ & $0.000000e+00$ & $0.000000e+00$ \\
& $-nan$ & $-nan$ & $-nan$ & $-nan$ \\
& $-nan$ & $-nan$ & $-nan$ & $-nan$ \\
& $-nan$ & $-nan$ & $-nan$ & $-nan$ \\
\hline
$1.000000e-02$ & $0.000000e+00$ & $0.000000e+00$ & $0.000000e+00$ & $0.000000e+00$ \\
& $-nan$ & $-nan$ & $-nan$ & $-nan$ \\
& $-nan$ & $-nan$ & $-nan$ & $-nan$ \\
& $-nan$ & $-nan$ & $-nan$ & $-nan$ \\
\hline
$1.000000e-03$ & $0.000000e+00$ & $1.008556e-02$ & $0.000000e+00$ & $0.000000e+00$ \\
& $-nan$ & $5.344376e-03$ & $-nan$ & $-nan$ \\
& $-nan$ & $6.545497e-03$ & $-nan$ & $-nan$ \\
& $-nan$ & $8.450201e-03$ & $-nan$ & $-nan$ \\
\hline
$1.000000e-04$ & $0.000000e+00$ & $1.057279e-02$ & $7.947876e-05$ & $1.028315e-04$ \\
& $-nan$ & $5.619152e-03$ & $4.532207e-05$ & $5.768326e-05$ \\
& $-nan$ & $6.882027e-03$ & $5.550798e-05$ & $7.064727e-05$ \\
& $-nan$ & $8.884659e-03$ & $7.166049e-05$ & $9.120524e-05$ \\
\hline
\end{tabular}


\begin{tabular}{ |l|l|l|l|l| }
\hline
\multicolumn{5}{|c|}{$\mu = 1.000000e-03, p(\rho) = 1.000000e+00\rho^{1.4}$} \\
\hline
$\tau\setminus h$ & $0.1$ & $0.01$ & $0.001$ & $0.0001$\\
\hline
$1.000000e-01$ & $8.178228e+00$ & $0.000000e+00$ & $0.000000e+00$ & $0.000000e+00$ \\
& $3.586242e+00$ & $-nan$ & $-nan$ & $-nan$ \\
& $4.392231e+00$ & $-nan$ & $-nan$ & $-nan$ \\
& $5.670346e+00$ & $-nan$ & $-nan$ & $-nan$ \\
\hline
$1.000000e-02$ & $0.000000e+00$ & $0.000000e+00$ & $0.000000e+00$ & $0.000000e+00$ \\
& $-nan$ & $-nan$ & $-nan$ & $-nan$ \\
& $-nan$ & $-nan$ & $-nan$ & $-nan$ \\
& $-nan$ & $-nan$ & $-nan$ & $-nan$ \\
\hline
$1.000000e-03$ & $0.000000e+00$ & $1.490782e-02$ & $7.002255e-03$ & $0.000000e+00$ \\
& $-nan$ & $5.501182e-03$ & $2.355502e-03$ & $-nan$ \\
& $-nan$ & $6.737545e-03$ & $2.884889e-03$ & $-nan$ \\
& $-nan$ & $8.698133e-03$ & $3.724376e-03$ & $-nan$ \\
\hline
$1.000000e-04$ & $0.000000e+00$ & $9.529754e-03$ & $7.506078e-04$ & $6.810461e-04$ \\
& $-nan$ & $3.967035e-03$ & $2.515955e-04$ & $2.277221e-04$ \\
& $-nan$ & $4.858606e-03$ & $3.081403e-04$ & $2.789015e-04$ \\
& $-nan$ & $6.272433e-03$ & $3.978074e-04$ & $3.600602e-04$ \\
\hline
\end{tabular}


\begin{tabular}{ |l|l|l|l|l| }
\hline
\multicolumn{5}{|c|}{$\mu = 1.000000e-03, p(\rho) = 1.000000e+00\rho$} \\
\hline
$\tau\setminus h$ & $0.1$ & $0.01$ & $0.001$ & $0.0001$\\
\hline
$1.000000e-01$ & $2.933296e+00$ & $0.000000e+00$ & $1.164766e+01$ & $0.000000e+00$ \\
& $1.218120e+00$ & $-nan$ & $3.931305e+00$ & $-nan$ \\
& $1.491886e+00$ & $-nan$ & $4.814846e+00$ & $-nan$ \\
& $1.926016e+00$ & $-nan$ & $6.215939e+00$ & $-nan$ \\
\hline
$1.000000e-02$ & $0.000000e+00$ & $2.084837e-01$ & $0.000000e+00$ & $0.000000e+00$ \\
& $-nan$ & $6.355737e-02$ & $-nan$ & $-nan$ \\
& $-nan$ & $7.784156e-02$ & $-nan$ & $-nan$ \\
& $-nan$ & $1.004930e-01$ & $-nan$ & $-nan$ \\
\hline
$1.000000e-03$ & $0.000000e+00$ & $2.327463e-02$ & $2.267530e-02$ & $2.270036e-02$ \\
& $-nan$ & $7.196256e-03$ & $5.192404e-03$ & $5.178483e-03$ \\
& $-nan$ & $8.813578e-03$ & $6.359371e-03$ & $6.342321e-03$ \\
& $-nan$ & $1.137828e-02$ & $8.209912e-03$ & $8.187901e-03$ \\
\hline
$1.000000e-04$ & $0.000000e+00$ & $9.636776e-03$ & $2.236876e-03$ & $2.199280e-03$ \\
& $-nan$ & $3.542850e-03$ & $5.293092e-04$ & $5.154662e-04$ \\
& $-nan$ & $4.339088e-03$ & $6.482688e-04$ & $6.313146e-04$ \\
& $-nan$ & $5.601738e-03$ & $8.369114e-04$ & $8.150236e-04$ \\
\hline
\end{tabular}


\begin{tabular}{ |l|l|l|l|l| }
\hline
\multicolumn{5}{|c|}{$\mu = 1.000000e-03, p(\rho) = 1.000000e+01\rho^{1.4}$} \\
\hline
$\tau\setminus h$ & $0.1$ & $0.01$ & $0.001$ & $0.0001$\\
\hline
$1.000000e-01$ & $8.178228e+00$ & $0.000000e+00$ & $0.000000e+00$ & $0.000000e+00$ \\
& $3.586242e+00$ & $-nan$ & $-nan$ & $-nan$ \\
& $4.392231e+00$ & $-nan$ & $-nan$ & $-nan$ \\
& $5.670346e+00$ & $-nan$ & $-nan$ & $-nan$ \\
\hline
$1.000000e-02$ & $0.000000e+00$ & $0.000000e+00$ & $0.000000e+00$ & $0.000000e+00$ \\
& $-nan$ & $-nan$ & $-nan$ & $-nan$ \\
& $-nan$ & $-nan$ & $-nan$ & $-nan$ \\
& $-nan$ & $-nan$ & $-nan$ & $-nan$ \\
\hline
$1.000000e-03$ & $0.000000e+00$ & $1.490782e-02$ & $7.002255e-03$ & $0.000000e+00$ \\
& $-nan$ & $5.501182e-03$ & $2.355502e-03$ & $-nan$ \\
& $-nan$ & $6.737545e-03$ & $2.884889e-03$ & $-nan$ \\
& $-nan$ & $8.698133e-03$ & $3.724376e-03$ & $-nan$ \\
\hline
$1.000000e-04$ & $0.000000e+00$ & $9.529754e-03$ & $7.506078e-04$ & $6.810461e-04$ \\
& $-nan$ & $3.967035e-03$ & $2.515955e-04$ & $2.277221e-04$ \\
& $-nan$ & $4.858606e-03$ & $3.081403e-04$ & $2.789015e-04$ \\
& $-nan$ & $6.272433e-03$ & $3.978074e-04$ & $3.600602e-04$ \\
\hline
\end{tabular}


\begin{tabular}{ |l|l|l|l|l| }
\hline
\multicolumn{5}{|c|}{$\mu = 1.000000e-03, p(\rho) = 1.000000e+01\rho$} \\
\hline
$\tau\setminus h$ & $0.1$ & $0.01$ & $0.001$ & $0.0001$\\
\hline
$1.000000e-01$ & $0.000000e+00$ & $0.000000e+00$ & $0.000000e+00$ & $0.000000e+00$ \\
& $-nan$ & $-nan$ & $-nan$ & $-nan$ \\
& $-nan$ & $-nan$ & $-nan$ & $-nan$ \\
& $-nan$ & $-nan$ & $-nan$ & $-nan$ \\
\hline
$1.000000e-02$ & $0.000000e+00$ & $0.000000e+00$ & $0.000000e+00$ & $0.000000e+00$ \\
& $-nan$ & $-nan$ & $-nan$ & $-nan$ \\
& $-nan$ & $-nan$ & $-nan$ & $-nan$ \\
& $-nan$ & $-nan$ & $-nan$ & $-nan$ \\
\hline
$1.000000e-03$ & $0.000000e+00$ & $1.064085e-02$ & $0.000000e+00$ & $0.000000e+00$ \\
& $-nan$ & $5.056446e-03$ & $-nan$ & $-nan$ \\
& $-nan$ & $6.192856e-03$ & $-nan$ & $-nan$ \\
& $-nan$ & $7.994943e-03$ & $-nan$ & $-nan$ \\
\hline
$1.000000e-04$ & $0.000000e+00$ & $1.142668e-02$ & $3.327649e-04$ & $3.152690e-04$ \\
& $-nan$ & $5.099298e-03$ & $1.456042e-04$ & $1.490651e-04$ \\
& $-nan$ & $6.245339e-03$ & $1.783280e-04$ & $1.825667e-04$ \\
& $-nan$ & $8.062698e-03$ & $2.302204e-04$ & $2.356927e-04$ \\
\hline
\end{tabular}


\begin{tabular}{ |l|l|l|l|l| }
\hline
\multicolumn{5}{|c|}{$\mu = 1.000000e-03, p(\rho) = 1.000000e+02\rho^{1.4}$} \\
\hline
$\tau\setminus h$ & $0.1$ & $0.01$ & $0.001$ & $0.0001$\\
\hline
$1.000000e-01$ & $8.178228e+00$ & $0.000000e+00$ & $0.000000e+00$ & $0.000000e+00$ \\
& $3.586242e+00$ & $-nan$ & $-nan$ & $-nan$ \\
& $4.392231e+00$ & $-nan$ & $-nan$ & $-nan$ \\
& $5.670346e+00$ & $-nan$ & $-nan$ & $-nan$ \\
\hline
$1.000000e-02$ & $0.000000e+00$ & $0.000000e+00$ & $0.000000e+00$ & $0.000000e+00$ \\
& $-nan$ & $-nan$ & $-nan$ & $-nan$ \\
& $-nan$ & $-nan$ & $-nan$ & $-nan$ \\
& $-nan$ & $-nan$ & $-nan$ & $-nan$ \\
\hline
$1.000000e-03$ & $0.000000e+00$ & $1.490782e-02$ & $7.002255e-03$ & $0.000000e+00$ \\
& $-nan$ & $5.501182e-03$ & $2.355502e-03$ & $-nan$ \\
& $-nan$ & $6.737545e-03$ & $2.884889e-03$ & $-nan$ \\
& $-nan$ & $8.698133e-03$ & $3.724376e-03$ & $-nan$ \\
\hline
$1.000000e-04$ & $0.000000e+00$ & $9.529754e-03$ & $7.506078e-04$ & $6.810461e-04$ \\
& $-nan$ & $3.967035e-03$ & $2.515955e-04$ & $2.277221e-04$ \\
& $-nan$ & $4.858606e-03$ & $3.081403e-04$ & $2.789015e-04$ \\
& $-nan$ & $6.272433e-03$ & $3.978074e-04$ & $3.600602e-04$ \\
\hline
\end{tabular}


\begin{tabular}{ |l|l|l|l|l| }
\hline
\multicolumn{5}{|c|}{$\mu = 1.000000e-03, p(\rho) = 1.000000e+02\rho$} \\
\hline
$\tau\setminus h$ & $0.1$ & $0.01$ & $0.001$ & $0.0001$\\
\hline
$1.000000e-01$ & $0.000000e+00$ & $0.000000e+00$ & $0.000000e+00$ & $0.000000e+00$ \\
& $-nan$ & $-nan$ & $-nan$ & $-nan$ \\
& $-nan$ & $-nan$ & $-nan$ & $-nan$ \\
& $-nan$ & $-nan$ & $-nan$ & $-nan$ \\
\hline
$1.000000e-02$ & $0.000000e+00$ & $0.000000e+00$ & $0.000000e+00$ & $0.000000e+00$ \\
& $-nan$ & $-nan$ & $-nan$ & $-nan$ \\
& $-nan$ & $-nan$ & $-nan$ & $-nan$ \\
& $-nan$ & $-nan$ & $-nan$ & $-nan$ \\
\hline
$1.000000e-03$ & $0.000000e+00$ & $0.000000e+00$ & $0.000000e+00$ & $0.000000e+00$ \\
& $-nan$ & $-nan$ & $-nan$ & $-nan$ \\
& $-nan$ & $-nan$ & $-nan$ & $-nan$ \\
& $-nan$ & $-nan$ & $-nan$ & $-nan$ \\
\hline
$1.000000e-04$ & $0.000000e+00$ & $1.522653e-02$ & $1.142716e-04$ & $0.000000e+00$ \\
& $-nan$ & $7.551270e-03$ & $5.727346e-05$ & $-nan$ \\
& $-nan$ & $9.248379e-03$ & $7.014538e-05$ & $-nan$ \\
& $-nan$ & $1.193961e-02$ & $9.055730e-05$ & $-nan$ \\
\hline
\end{tabular}




% Для тестового запуска используйте:
% \begin{tabular}{ |l|l|l|l|l| }
\hline
\multicolumn{5}{|c|}{$\mu = 0.1, p(\rho) = 1\rho$} \\
\hline
$\tau\setminus h$ & $0.1$ & $0.01$ & $0.001$ & $0.0001$\\
\hline
$0.1$ & $1.119882e+00$ & $1.625199e+00$ \\
& $6.826320e-01$ & $7.242010e-01$ \\
& $1.079336e+00$ & $1.145062e+00$ \\
& $1.000000e-02$ & $1.000000e-03$ \\
\hline
$1.000000e-02$ & $3.505647e-01$ & $1.675353e-01$ \\
& $1.862471e-01$ & $7.960856e-02$ \\
& $2.944825e-01$ & $1.258722e-01$ \\
& $1.000000e-03$ & $1.000000e-04$ \\
\hline
\end{tabular}


\begin{tabular}{ |l|l|l|l|l| }
\hline
\multicolumn{5}{|c|}{$\mu = 1.000000e-01, p(\rho) = 1.000000e+01\rho$} \\
\hline
$\tau\setminus h$ & $0.1$ & $0.01$ & $0.001$ & $0.0001$\\
\hline
$1.000000e-01$ & $0.000000e+00$ & $3.288957e+00$ \\
& $-nan$ & $1.781050e+00$ \\
& $-nan$ & $2.816087e+00$ \\
& $1.000000e-02$ & $1.000000e-03$ \\
\hline
$1.000000e-02$ & $2.599785e+01$ & $1.802300e-02$ \\
& $2.025212e+01$ & $6.928088e-03$ \\
& $3.202142e+01$ & $1.095427e-02$ \\
& $1.000000e-03$ & $1.000000e-04$ \\
\hline
\end{tabular}




\section{Анализ результатов}

\subsection{Сходимость метода}

Из представленных таблиц можно сделать следующие выводы:

\begin{itemize}
    \item При уменьшении шагов сетки $h$ и $\tau$ невязки уменьшаются, что свидетельствует о сходимости численного метода
    \item Для различных значений параметра $\mu$ наблюдается различная скорость сходимости
    \item При больших значениях $C_\rho$ требуются более мелкие сетки для достижения приемлемой точности
\end{itemize}

\subsection{Влияние параметров}

\textbf{Параметр $\mu$ (вязкость):}
\begin{itemize}
    \item При уменьшении $\mu$ метод становится менее устойчивым
    \item Требуется уменьшение шагов сетки для сохранения точности
\end{itemize}

\textbf{Параметр $C_\rho$ (коэффициент в уравнении состояния):}
\begin{itemize}
    \item Большие значения $C_\rho$ приводят к более жёстким системам
    \item Может потребоваться адаптивный выбор шагов
\end{itemize}

\textbf{Тип уравнения состояния:}
\begin{itemize}
    \item $p(\rho) = C_\rho \rho$ -- линейная зависимость
    \item $p(\rho) = C_\rho \rho^{1.4}$ -- политропный газ
\end{itemize}

\section{Заключение}

Разработанный численный метод позволяет эффективно решать задачи газовой динамики с различными параметрами. Результаты показывают хорошую сходимость при правильном выборе шагов сетки.

\end{document}
